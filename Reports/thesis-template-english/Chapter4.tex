\chapter{Results and Discussion}\label{ch:results_discussion}


\section{Results}\label{results}
In this section we present the results of our experiments. In the first part we will examine the overall results regarding reductive evolution before examining some other interesting findings in more detail.

\subsection{Condition Analysis}
In this section we will examine which conditions, if any, lead to a reduced genome. Figure~%TODO add figure showing genome size
presents the main findings regarding genome size. In the Figure, the blue line %TODO make sure the line is actually the blue one
represents the control condition and the other colors show the changed conditions: mutation up/down, selection up/down, and population up/down. As can be seen from the figure, 





\section{Discussion}\label{discussion}

\subsection{Relation to Real-World Biological Entities}

\subsection{Limitations of Results}\label{limitations}
One limitation to consider is that only one parameter varied per condition. It may be possible that it is only under a combination of conditions (e.g. low selection \textit{and} high mutation rates) does reductive evolution occur. 

Another limitation is that the environments did not vary in our experiments. This could potentially have a large effect on robustness and evolvability, which are strong influencers of reductive evolution. 