
\chapter{Conclusions and Future Work}\label{ch:05conclusion}

In this chapter, the overall conclusions drawn from the work are summarized. Additionally, some limitations of the experiments are discussed, and a few possibilities for further research are highlighted.

\section{Conclusions}
There were four conditions in which, at the conclusion of these experiments in in silico experimental evolution, the genome size was smaller than the control condition (see Figure~\ref{fig:genome_size}): $\mu_+$, $k_-$, $N_+$, and surprisingly $\mu_-$. 

Overall, direct selection did not seem to have played much of a role in reducing the genome, again in agreement with Batut et al.~\cite{Batut.2014}. The mean population genome size in the $k_+$ condition ended up being larger than the control condition, likely because of the accumulation of non-coding bases, and for the $k_-$ condition, the results were found to not be statistically significant. 

Changing the mutation rate had a noticeable impact on genome size, however, largely through the introduction of non-coding sequences ($\mu_+$) or through the loss of non-coding bases ($\mu_-$). The large increase in the level of mutational robustness seen in the $\mu_+$ condition seemed able to overcome the proposed ``complexity ratchet'' of Liard et al.~\cite{Liard.2018}, increasing robustness by 230\% (see Table~\ref{table:robustness_means_and_std_dev}) while reducing the genome by about 1.1\% (see Table~\ref{table:genome_size_stats}). The steady selection pressure in the $\mu_-$ condition allowed selection to ``keep up'' with the random insertions of non-coding bases, eventually removing them and reducing the overall population mean genome size by 2.76\%. 

Though it is very unlikely that any single factor can single-handedly lead to reductive genome evolution, based on the results of these experiments, the single largest driver of reductive genome evolution in these scenarios does seem to be effective population size ($N_e$), which was able to reduce the population mean genome size by 7.5\% by the last 50,000 generations (see Table~\ref{table:genome_size_stats}). As discussed by Batut et al.~\cite{Batut.2014}, the increased effective population size appears to increase the efficacy of selection to reduce the genome via the removal of non-coding bases and superfluous genes, while at the same time increasing the overall fitness. The average size of coding sequences grew smaller, there were fewer non-coding bases, and the mean number of functional genes increased almost as much as in the $k_+$ condition, all without the requisite fitness loss. In other words, the genomes become more dense but fitness was not affected, strong characteristics of effective reductive genome evolution. 

\section{Relation to Real-World Organisms}

This thesis was begun with the goal of examining the factors that may drive reductive evolution, with the overall intention of relating this to real-world examples. The marine cyanobacteria \textit{Prochlorococcus} has large effective population sizes and high adaptability, which has resulted in a large number of ecotypes which exist at differing depths in the oligotrophic oceans. Some possible reasons for this large reduction in genome size for \textit{Prochlorococcus} were examined in Batut et al.\cite{Batut.2014} and there, the most robust explanation for genomic reduction in \textit{Prochlorococcus} was found to be adaptation to the environment, as the benefit of losing specific cellular machinery for multiple different environments likely outweighed the cost of maintaining them. The experimental results here seem to confirm this hypothesis, as the highly stable environment seems to have allowed the artificial organisms to shed superfluous genes over time, all the while gaining fitness. The larger effective population size of the $N_+$ condition clearly showed the largest increase in reductive evolution

The fittest organisms overall were those in the $k_-$ condition. It was hypothesized in Knibbe et al.~\cite{Knibbe2007} and Batut et al.~\cite{Batut.2013} that with decreased selection, the accompanying loss of non-coding bases (and thus a reduction in the genome size) might also cause an increase in fitness, and this was borne out by these experiments. Not only was the $k_-$ condition the condition with the best fitness, but it also saw the second largest decrease in non-coding bases. 
  
\section{Limitations and Future work}\label{limitations}
A number of limitations are associated with this work and they are briefly presented here. 
\subsection{Experimental Design Limitations}
Although it was the main design of these experiments, one limitation to consider is that only one parameter varied per condition. It seems reasonable to assume that under a combination of conditions, reductive evolution more reliably occurs. Since an increased mutation rate, decreased selection pressure, and an increased population size all lead to a reduced genome in the experiments, it may be worth testing whether a combination of the three would lead to an even greater reduction. 

In a similar vein, only one value for each changed condition was tested. For example, only a mutation rate of $\mu = 4e^{-7}$ was tested for the $\mu_+$ condition. Perhaps an even higher mutation rate might be enough to further reduce the number of genes, as predicted by Knibbe et al.~\cite{Knibbe2007} and \cite{Liard.2018}. Their mutation rates were $2.1e^{-5}$ and $1e^{-3}$ respectively, roughly $3.9$ and $13.65$ times higher than the rate found here. The testing of more extreme values for each parameter may provide a fuller picture of the balance between selection, mutation rates, population sizes, and reductive genome evolution. As further evidence of the potentially limiting rate of mutation used in these experiments, when examining the work of Knibbe et al. regarding ``successful'' genomes requiring mostly neutral or beneficial offspring (see Section~\ref{related_work}), the link between a greater increase in the mutation rate (and thus mutational robustness) and genome reduction bears further research. 

Another limitation is that the environments did not vary in these experiments. Aevol includes the ability to vary the environment natively, so future experiments could vary the environment over time to see if doing so would increase or decrease robustness or evolvability, which are strong drivers of reductive evolution~\cite{Batut.2013}. 

\subsection{Limitations of the Aevol Model}
Aevol as a modeling software is limited in that, like all models, it relies on several simplifications. For example, the population sizes tested here, even in the population up condition where $N = 4096$, are still much smaller than would be found in real world bacterial populations. It has been estimated that some reduced \textit{Prochlorococcus} populations may have an effective population size of $N_e \approxeq 10^9$ \cite{kashtan2014single}. It could be that in order to properly judge the effects of Muller's ratchet under differing conditions, for example, much larger population sizes are needed. However, the length of time required for running such a simulation in Aevol even on a decent cluster might make this intractable, as quadrupling the population size for the $N_+$ condition also greatly increased the run-time of these experiments. Currently, running the simulation with even $4096$ organisms for 500,000 generations required over five days of computation time on a 96-core cluster. Future versions of Aevol may see more optimization, but simulating population sizes on the order of magnitude of \textit{Prochlorococcus}' estimated effective population size currently remains a daunting task.  

Another limitation of the Aevol model is that, as mentioned in Section~\ref{res:metabolic_error_and_fitness}, non-coding DNA has no associated fitness costs in Aevol. In real-world bacteria, replicating non-coding DNA has a certain cost, as the resources required for these bases must still be procured and a certain amount of energy expended for the replication. Nevertheless, there is indirect selection pressure associated with non-coding DNA, as Knibbe et al. showed, because ``varying the amount of nonfunctional DNA would not change the immediate fitness of the organism but would rather modify the chance that its offspring retain it''~\cite{Knibbe2007}. 

Lastly, Aevol also relies on the simplification that its space of ``biological functions'' are a one-dimensional function, when the real-world space is obviously much more complicated. Mapping real-world biological functions to a one-dimensional space is difficult enough in and of itself, but this simplification also effectively disallows the use of gene networks, which may be an important factor in determining which genes are lost in reductive evolution~\cite{wilcox2003consequences}. Similarly, Aevol is also limited in its representation of pleiotropy (a single gene influencing two or more phenotypic traits). In Aevol, a protein can perform several functions but these functions must necessarily be functionally close together, as their requisite bases are close on the possibility axis. In real organisms, however, a single protein can perform very different functions; this may influence gene loss events as well, given the typically dense nature of bacterial genomes, \textit{Prochlorococcus} being no exception.