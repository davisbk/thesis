
\chapter{Conclusions and Future Work}\label{ch:05conclusion}
%TODO Remove chapter 5 summary?

In this chapter, the overall conclusions drawn from the work are summarized. Additionally, some limitations of the experiments are discussed, and a few possibilities for further research are highlighted.

\section{Conclusions}
%TODO Complete Conclusions
At generation 500,000 there were three conditions in which the genome size was lower than the control condition (see Figure~\ref{fig:genome_size}): $\mu_+$, $k_-$, and surprisingly $\mu_-$. The predictions for these conditions having the largest amount of reductive evolution were described in Section~\ref{sec:expected_results} and are summarized again here. 

\section{Relation to Real-World Organisms}

This thesis was begun with the goal of examining the factors that may drive reductive evolution, with the overall intention of relating this to real-world examples. The marine cyanobacteria \textit{Prochlorococcus} has large effective population sizes and high adaptability, which has resulted in a large number of ecotypes which exist at differing depths in the oligotrophic oceans. Some possible reasons for this large reduction in genome size for \textit{Prochlorococcus} were examined in Batut et al.\cite{Batut.2014} and there, the most robust explanation for genomic reduction in \textit{Prochlorococcus} was found to be adaptation to the environment, as the benefit of losing specific cellular machinery for multiple different environments likely outweighed the cost of maintaining them. The experimental results here seem to confirm this hypothesis, as the highly stable environment seems to have allowed the artificial organisms to shed superfluous genes over time while gaining fitness. 

The fittest organisms overall were those in the $k_-$ condition. It was hypothesized in Knibbe et al.~\cite{Knibbe2007} and Batut et al.~\cite{Batut.2013} that with decreased selection, the accompanying loss of non-coding bases (and thus a reduction in the genome size) might also cause an increase in fitness, and this was borne out by these experiments. Not only was the $k_-$ condition the condition with the best fitness, but it also saw the second largest decrease in non-coding bases. 
  
\section{Limitations and Future work}\label{limitations}
A number of limitations are associated with this work and they are briefly presented here. 
\subsection{Parameters}
Although it was the main design of these experiments, one limitation to consider is that only one parameter varied per condition. It may be possible that it is only under a combination of conditions that reductive evolution more reliably occurs. Since both an increased mutation rate and decreased selection pressure lead to a reduced genome in the experiments, it may be worth testing whether a combination of the two would lead to an even greater reduction. 

In a similar vein, only one value for each changed condition was tested. For example, only a mutation rate of $\mu = 4e^{-7}$ was tested for the $\mu_+$ condition. Perhaps an even higher mutation rate, such as $\mu_+ = 1e^{-3}$, might be enough to reduce the number of genes, as predicted by Knibbe et al.~\cite{Knibbe2007} and \cite{Liard.2018}. Their mutation rates were $2.1e^{-5}$ and $1e^{-3}$, roughly $3.9$ and $13.65$ times higher than the rate found here. The testing of more extreme values for each parameter may provide a fuller picture of the balance between selection, mutation rates, population sizes, and reductive genomes. 

\subsection{Environments}
Another limitation is that the environments did not vary in these experiments. Aevol includes the ability to vary the environment natively, so future experiments could vary the environment over time to see if doing so would increase or decrease robustness or evolvability, which are strong drivers of reductive evolution~\cite{Batut.2013}. 

\subsection{Limitations of the Aevol Model}
Aevol as a modeling software is limited in that, like all models, it relies on several simplifications. For example, the population sizes tested here, even in the population up condition where $N = 4096$, are still much smaller than would be found in real world populations. It has been estimated that some reduced \textit{Prochlorococcus} populations may have an effective population size of $N_e \approxeq 10^9$ \cite{kashtan2014single}. It could be that in order to properly judge the effects of Muller's ratchet under differing conditions, much larger population sizes are needed. However, the length of time required for running such a simulation in Aevol even on a decent cluster might make this intractable, as quadrupling the population size also quadrupled the run-time of these experiments. Currently, running the simulation with even $4096$ organisms required over two weeks of computation time on a 16-core machine. 

Aevol also relies on the simplification that its space of ``biological functions'' are a one-dimensional function, when the real-world space is obviously much more complicated. Mapping real-world biological functions to a one-dimensional space is difficult enough in and of itself, but this simplification also effectively disallows the use of gene networks, which may be an important factor in determining which genes are lost in reductive evolution~\cite{wilcox2003consequences}. Lastly and related to the difficulties of properly mapping the protein space, Aevol also is limited in its representation of pleiotropy (a single gene influencing two or more phenotypic traits). In Aevol, a protein can perform several functions but these functions must necessarily be functionally close together, as they are close on the possibility axis. In real organisms, however, a protein can perform very different functions; this may influence gene loss events as well, given the typically dense nature of bacterial genomes (and \textit{Prochlorococcus} being no exception).