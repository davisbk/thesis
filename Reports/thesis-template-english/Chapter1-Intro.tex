\pagenumbering{arabic}%DO NOT REMOVE THIS
\chapter{Introduction and Background}\label{ch:intro}
Some species of bacteria have experienced reductive evolution over the course of millions of years, and this reduction in their genome has lead to a loss of genes, regulatory abilities, etc. The reasons for this are an area of active research %TODO insert citation for current research
, and one such field of research is \textit{in silico evolution}. In this method, organisms and their evolution over thousands or millions of generations are simulated in software. In this manner, one can control and evaluate every aspect of their evolution over time, potentially leading to a greater understanding of the factors that lead to specific effects on the genome. The in silico tool \textit{aevol} is one such platform which realistically models bacterial genomes and evolution, allowing one to draw conclusions about their real-world counterparts. In the following thesis, we present the results of our experiments in artificial evolution which aim to identify and evaluate several factors which lead to a reduced genome in simulated bacteria using the aevol platform. 

\section{Problem Statement}


\subsection{Report outline}
This chapter serves as the introduction to the thesis and the research problem we are facing. In Chapter~\ref{ch:background}, we provide some necessary background information on in silico evolution in general and aevol in particular. Chapter~\ref{ch:approach} describes our experimental setup. Chapter~\ref{ch:results_discussion}
provides the results and analysis of the experiments, and Chapter~\ref{ch:conclusion} gives our conclusions. 

In this section we provide some general background information about in silico evolution and aevol, the artificial evolution platform used in our experiments. 

